\documentclass[a4paper,openright,12pt]{article}
\usepackage[utf8]{inputenc}
\usepackage{graphicx} 
\usepackage{subfigure}
\usepackage[mathscr]{eucal}
\usepackage{titling}
\usepackage{float}
\usepackage{amsmath}
\usepackage{afterpage}
\usepackage{vmargin}
\usepackage[spanish]{babel}
\usepackage{eurosym} 
\usepackage{multirow} 
\usepackage{cite}
\usepackage{url}

\setpapersize{A4}	   %  DIN A4
\setmargins{3cm}    % margen izquierdo
{3.5cm}                     % margen superior
{15cm}                       % anchura del texto
{22.5cm}                   % altura del texto
{10pt}                         % altura de los encabezados
{1cm}                         % espacio entre el texto y los encabezados
{0pt}                           % altura del pie de página
{2cm}                         % espacio entre el texto y el pie de página

\begin{document}

\begin{titlepage}

\begin{center}
\vspace*{-1in}
\begin{figure}[htb]
\begin{center}
\includegraphics[width=8cm]{udc.eps}
\end{center}
\end{figure}

\vspace*{1in}
PROGRAMACIÓN DE SISTEMAS 21/22 Q1\\
Icono de la aplicación\\
\vspace*{1in}
\begin{Large}
\textbf{Título del Proyecto} \\
\end{Large}

\vspace*{3in}

\begin{large}
\raggedleft
\textbf{Autores:}Miguel Blanco Godón \\
Blanca Fernández Martín\\
Laura Cabezas González\\
\textbf{Fecha:}\textit{A Coruña, 7 de Octubre de 2021}\\
\end{large}

\end{center}
\end{titlepage} 

\newpage

\addtocontents{toc}{\hspace{-7.5mm} \textbf{Capítulos}}
\addtocontents{toc}{\hfill \textbf{Página} \par}
\addtocontents{toc}{\vspace{-2mm} \hspace{-7.5mm} \hrule \par}

\pagenumbering{empty}

\tableofcontents

\vspace{5cm}

\begin{flushright}
\begin{table}[hbtp]
\begin{center}

\caption{Tabla de versiones.}
\label{tabla:versiones}
\small
\vspace{1ex}

\begin{tabular}{|c|c|l|}
\hline
Versión & Fecha & Autor \\
\hline \hline
0.1 & 07/10/2021 & Grupo Q1.1 \\ \hline
x & y & \\ \hline
x & y & \\ \hline
\end{tabular}

\end{center}
\end{table}
\end{flushright}


\newpage
\pagenumbering{arabic}


%%%%%%%
%%%%%%%
\section{Introducción}\label{cap.introduccion}

%%
\subsection{Objetivos}
El objetivo es la elaboración de una app de tipo agenda para organizar el estudio. Se podrán fijar fechas de entrega y exámenes de forma que se acceda a un calendario al que se puedan añadir eventos. También se podrá almacenar una copia del horario de clase en la aplicación, con información adicional de cada asignatura, como, por ejemplo, el aula en la que se imparte cada clase o el profesor. Se podrá entrar a la aplicación autenticándose mediante usuario y contraseña, y gracias a esto los usuarios registrados podrán organizarse en grupos, para sincronizar su forma de gestionar el estudio y repartir tareas entre ellos, de forma que cada miembro del grupo pueda marcar el estado en el que se encuentra la parte de la tarea en la que esta trabajando (si está trabajando todavía en ella, si está terminada, etc.) y el resto de integrantes puedan verlo. La aplicación también podrá acceder al reloj y podrá fijar recordatorios o alarmas en función de la hora, de forma que pueda avisar al usuario de que un día en concreto a partir de una hora determinada ha planeado que se tiene que poner a estudiar. Desde la aplicación también será posible compartir los documentos de los trabajos en grupo en la nube, y descargarlos para tener una copia y acceder a ellos sin conexión.
Se utilizará GitHub para el control de versiones de nuestra app. \cite{misc-git}
%%
\subsection{Motivación}
La buena gestión del tiempo es un aspecto básico para la consecución de un curso escolar. Por ese motivo necesitamos un mecanismo de gestión que nos ayude a mejorar nuestra producción académica con el menor esfuerzo posible, a través de un sistema sencillo, fácil de manejar, con control automático y centralizado que nos informe sobre los detalles de las diferentes tareas. 
Esta gestión se vuelve crítica cuando las tareas a realizar son grupales, especialmente si los integrantes tienen diferentes horarios. En este caso, un sistema de sincronización de tareas con recursos compratidos el cual monitorizase y notificara las tareas hechas y venideras a los diversos integrantes sería muy útil.
%%
\subsection{Trabajo relacionado}
Aplicaciones relacionadas con partes de nuestra app:
- TimeTune es una aplicación de gestión de tiempo y planificador de horarios \cite{misc-url1}

Reminder es una apps para recordatorios, alarmas y tareas. \cite{misc-url2}

Exam-countdown es una app para realizar un seguimiento de los exámenes y pruebas fechas.\cite{misc-url3}

Además también nos inspira mucho a la hora de saber las entregas la propia interfaz de moodle, ya que nos va indicando las entregas y su fecha límite.

%%%%%%%
%%%%%%%
\section{Análisis de requisitos}

%%
\subsection{Funcionalidades}
Las funcionalidades principales serán:
\begin{itemize}
\item \textbf{Fijar fechas en el calendario:} desde la aplicación habrá una forma de establecer citas que se crearán en un calendario. Se podrá fijar la fecha, el nombre del evento, y alguna nota informativa a mayores. La aplicación llevará una agenda donde se mostrarán ordenadamente los eventos creados y sus fechas.

\item \textbf{Almacenar el horario de clases:} se podrá crear un horario semanal introduciendo las asignaturas y algún atributo a mayores, por ejemplo, el aula en la que se imparte o el material que es necesario llevar. El horario será visible nada más abrir la aplicación.

\item \textbf{Autenticación:} el usuario podrá identificarse al entrar en la app, de forma que pueda acceder a sus datos desde otros dispositivos. Además, la autenticación del usuario verifica su identidad frente a otros usuarios a la hora de trabajar en grupos.

\item \textbf{Grupos de trabajo:} se podrán establecer tareas conjuntas para los integrantes de un grupo, y cada integrante podrá ir actualizando el avance de su trabajo de forma pública para el resto de sus compañeros.

\item \textbf{Notificaciones de estudio:} el usuario podrá fijar el tiempo diario que le quiera dedicar a cada tarea y organizarlo en el día. La aplicación almacenará esta información y emitirá notificaciones a las horas establecidas a modo de recordatorio.

\item \textbf{Apuntes en la nube:} el usuario podrá acceder a archivos que hayan compartido otros usuarios de a los que pertenezca. Será posible descargar estos documentos para poder acceder sin conexión. 
\end{itemize}

%%
\subsection{Prioridades}
El objetivo principal de esta aplicación es implementar de forma simple las funcionalidades de una agenda, aprovechando las comodidades que puede aportar un telefono movil. Es por esto que las prioridades serán las tareas más cercanas a la funcionalidad de la agenda, es decir, fijar las fechas de los eventos importantes en un calendario, mostrar los eventos proximos, mantener una copia facilmente accesible del horario semanal y el sistema de notificaciones para el estudio diario. 



%%%%%%%
%%%%%%%
\section{Planificación inicial}

%%
\subsection{Iteraciones}
Se harían cuatro iteraciones incrementales. En la primera, se intentará que la aplicación tenga su funcionalidad básica: horario, gestión de calendario y usuarios.
En la segunda iteración se haría la extensión multiusuario y notificación: notificaciones, creación de grupos y sincronización básica de tareas. También se corregirían errores de la iteración anterior.
En la tercera iteración se añadirá el repositorio de datos común y la gestión comunal de tareas. También se corregirán los errores de la iteración anterior.
En la iteración final, se corregirán los fallos de las iteraciones anteriores y se hará la integración final de conjunto.

%%
\subsection{Responsabilidades}
Se llevará a cabo la implementación de tres funcionalidades simultaneamente, para que todos los miembros del grupo puedan trabajar simultaneamente sin tener que esperar a por el trabajo de los demás. Las primeras funcionalidades a implementar se repartirán tal que 
la creación eventos en el calendario será implementada por Blanca Fernandez, el horario semanal por Laura Cabezas y la autenticación por Miguel Blanco. El resto de funcionalidades se repartirán más adelante cuando se hayan adquirido los conocimientos necesarios para estimar correctamente el esfuerzo requerido por cada una.
%%
\subsection{Hitos}
Se realizará una entrega por cada funcionalidad implementada, más otras dos a mayores, al final de cada iteración, donde se solucionarán los detalles de integración de las partes en las que se dividió el trabajo. Por lo tanto, habrá tres hitos iniciales para las tres funcionalidades ya repartidas, y un cuarto hito donde se mostrará una primera versión funcional de la aplicación, aunque con funcionalidades reducidas. Más adelante, volverá a haber un hito por cada funcionalidad secundaria que se implemente, y finalmente, la entrega final de la aplicación completa. 
En cuanto a los test, en cada hito de una funcionalidad aislada se creará un test para esa funcionalidad, y para el hito que compruebe la integración de las partes, un test común para las funcionalidades que estén implementadas hasta ese momento.
%%
\subsection{Incidencias}
Se completará esta sección cuando se haya llegado a un grano suficientemente fino en la planificación del proyecto y se conozcan más detalles sobre la implementación.
Cuando estos detalles sean conocidos se clasificarán las posibles incidencias segun su gravedad y se poporcionará una solución acorde. Los diferentes grados serán incidencias leves (bugs), incidencias de grado medio (fallo en una funcionalidad/caso de uso), o incidencias graves (a nivel de toda la aplicación/arquitectura/caso de uso no satisfacible).

%%%%%%%
%%%%%%%
\section{Diseño}

%%
\subsection{Arquitectura}
Para esta aplicación vamos a usar el patrón de arquitectura MVC.

\subsection{Persistencia}
Dependiendo de la funcionalidad necesitaremos guardar unos datos distintos, por ejemplo, para guardar una evento en el calendario tendremos que guardar el nombre, fecha de entrega, etc. Y en el horario tendremos, por ejemplo, la asignatura y la clase donde se imparte.
\subsection{Vista}
La vista va depender de como vayamos implementando las funciones, en un principio contamos usar actividades, fragmentos, notificaciones, buttons, recyclerview, etc.
\subsection{Comunicaciones}
Necesitamos comunicaciones con servidores para la manutención de usuarios y con la API de Google para el calendario. También necesitaremos un proveedor de bases de datos.
\subsection{Sensores}
No aplica.
\subsection{Trabajo en background}
Sincronización de datos, gestión de eventos a efectos de notificación.




%%%%%%%
%%%%%%%




\bibliographystyle{pfc-fic}
\bibliography{biblio}
\addcontentsline{toc}{section}{Bibliografía}

\end{document}
